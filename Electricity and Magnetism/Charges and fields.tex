\documentclass{article}

\usepackage{array,epsfig}
\usepackage{amsmath}
\usepackage{amsfonts}
\usepackage{amssymb}
\usepackage{amsxtra}
\usepackage{amsthm}
\usepackage{mathrsfs}
\usepackage{color}
\usepackage{url}

\title{Electrostatics: Charges and Fields}
\author{Aayush Arya\\ (tomriddle257@gmail.com)}

\begin{document}
	\maketitle
	
	\section*{Charge: Definition and Properties}
	Charge is a property of elementary particles which causes them to experience electromagnetic force, whose strength depends on the magnitude of the charge itself.It comes in two flavors: positive and negative.\\

	\subsection*{Conservation}
	The amount of charge in an isolated system is a constant \textemdash which means that charge is locally conserved and can't be destroyed.
	$$ \sum_{i=1}^{n} q_i = constant$$
	 Even in case of pair-production, anti-particles have opposite charges (as well as parity and time flow) as their matter counterparts. The opposite charges of an electron and positron are exactly equal. If this weren't the case, the \textit{law of charge conservation} would be violated. \\
	
	\subsection*{Quantization}
	
	Electrical charges can only be multiples of $e$ \textemdash the charge of an electron. Up and down quarks have charges with values as fractions of $e$, \textit{quantum chromodynamics} suggests that it is impossible to liberate a single quark from a hadron.\\
	
	Charge of a proton happens to be equal to that of an electron \\ J.G. King (1960) edid an experiment to confirm it. He took ~17 grams of $H_2$ gas \textemdash which consists of 2 protons and 2 electrons \textemdash  compressed into a container electrically insulated from its surroundings. The gas was allowed to escape in a way that ions could not accompany.\\
	If the charge on a proton were to differ even by one part in $10^{20}$, the difference made by the absence of the ~$5\times 10^{24}$ hydrogen molecules would have made a difference of $5\times10^{4}$e which could've been detected. No such difference was observed.
	
	Also, during a proton decay (which hasn't been observed),
		$$ p^+ \rightarrow e^+ + \pi^0 $$
	if the charge of the positron was different from that of the proton, again charge conservation would be violated.
	
	\section*{Coulomb's Law}
	To study the interaction of between charged  particles, we treat as being point-sized. This is justified as high-energy physics experiments show that the charge of a proton does not extend appreciably over $10^{-15} m$.\\
	
	The force between two such particles is then given by \textit{Coulomb's law}
	\begin{equation}\textbf{F}_2 = \frac{1}{4\pi\epsilon_0}\frac{q_1q_2\hat{\textbf{r}}_{21}}{r_{21}}
	\end{equation}
	The $\hat{\textbf{r}}_{21}$ suggests that the direction of the force is along the line joining the two charges \textemdash it is a central force.\\
	
	1 esu or "electrostatic unit" is a Gaussian unit defined in a way that $k$ equals exactly one when  other quantities are measured in cgs. 1 Coulomb = $2.998 \times 10^9$ esu.
	
\end{document}