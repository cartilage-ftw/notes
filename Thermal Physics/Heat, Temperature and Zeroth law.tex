\documentclass{article}

\usepackage{array,epsfig}
\usepackage{amsmath}
\usepackage{amsfonts}
\usepackage{amssymb}
\usepackage{amsxtra}
\usepackage{amsthm}
\usepackage{mathrsfs}
\usepackage{color}
\usepackage{url}

\title{Heat, Zeroth Law and Temperature}
\author{Aayush Arya\\ (tomriddle257@gmail.com)}

\begin{document}
	
	\maketitle
	
	\section*{What is Heat?}
	Heat is strictly defined as \textit{energy in transit}. It's meaningless to say that a certain object contains $x$ amount of heat. Similarly, \textit{work} can be made sense of only as "mechanical energy in transit". Spontaneously, when placed in thermal contact, heat flows from a hotter object to a colder one. We note that the seemingly unrelated quantity, work done (W), can be converted into heat. \\
	
	"Heat capacity" is not the amount of heat a system can take, but rather the amount of heat (energy) needed to raise the system's temperature by one unit (1K). Mathematically, we can write
	$$ C = \frac{dQ}{dT}$$
	Specific heat capacity, on the other hand, is the amount of energy that must be supplied to a unit mass of a substance to raise its temperature by 1K.
	
	There are some complications when we try to measure C of a substance. Gases expand very quickly and while doing so, do some work on the surrounding. This is compensated by requiring extra energy to raise its temperature. Therefore,
	$$ C_p = \left(\frac{\partial Q}{\partial T}\right)_p $$
	is the heat capacity at constant pressure. Similarly,
	\begin{equation*}
	C_V = \left(\frac{\partial Q}{\partial T}\right)_V
	\end{equation*}
	 describes the heat capacity at constant volume.
	 
	 \section*{Temperature and Zeroth Law}
	 Two objects, say A and B, are said to be in "thermal contact" if they can exchange energy. When left in contact, heat flows from the hotter body to the colder body spontaneously. When the two attain the same final temperature $T_f$, the flow of heat from A $\rightarrow$ B becomes equal to that from B $\rightarrow$ A.\\
	 The two bodies are then said to be in \textbf{thermal equilibrium} with each other as the net flow of heat becomes zero. \\
	 
	 The \textbf{Zeroth law of thermodynamics} says that if two bodies, A and B are in thermal equilibrium with each other and B is in thermal equilibrium with a third body, say C, then A and C must also be in thermal equilibrium.\\
	 This is a profound fact because it tells us that we can assign a macroscopic property related to energy flow that can be universally compared. We name this property "temperature".\\
	 
	 \subsection*{Thermometers}
	 An instrument for measuring temperature can be made. The reading of a thermometer stops changing over time when the object being measured and the thermometer are in thermal equilibrium.\\
	 The first thermometer was made by Galileo which used water. Farenheit invented alcohol as well as mercury thermometers.\\
	 
	 Other instruments include resistance thermometers (which usually use Platinum due to high chemical resistance). Apart from Pt, doped Germaninum which shows high stability over thermal recycling.
	 Carbon sensors, $RuO_2$ (whose resistance goes up on cooling) are all used.\\
	 In cryogenics, the vapour pressure \textendash temperature relationship of liquid $He$ is very useful.
	 
	% \subsection{Problems with thermometers}
	 There are a few challenges with defining temperature using thermometer scales.
	 
	 \begin{enumerate}
	 	\item Firstly, to make very precise measurements, the heat capacity of the thermometer should be \textit{very low}.
	 	
	 	\item No substance gives a completely linear temperature scale. Mercury solidifies at low temperature and vaporizes at greater temperature.
	 	
	 	\item Theoretical calculations can be made using the ideal gas equation for gas but gases liquify at cryogenic temperatures and depart from ideal gas behavior.
	 \end{enumerate}
	 
	 
	 
\end{document}