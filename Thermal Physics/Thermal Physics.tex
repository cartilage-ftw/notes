\documentclass{article}

\usepackage[margin=1.0in]{geometry}

\usepackage{array,epsfig}
\usepackage{amsmath}
\usepackage{amsfonts}
\usepackage{amssymb}
\usepackage{amsxtra}
\usepackage{amsthm}
\usepackage{mathrsfs}
\usepackage{color}
\usepackage{url}

\def\dbar{{\mathchar'26\mkern-12mu d}}

\title{Thermal Physics}
\author{Aayush Arya\\ (tomriddle257@gmail.com)}

\begin{document}
	
	\maketitle
	
	%\tableofcontents
	%\newpage
	\section{What is Heat?}
	Heat is strictly defined as \textit{energy in transit}. It's meaningless to say that a certain object contains $x$ amount of heat. Similarly, \textit{work} can be made sense of only as "mechanical energy in transit". Spontaneously, when placed in thermal contact, heat flows from a hotter object to a colder one. We note that the seemingly unrelated quantity, work done (W), can be converted into heat. \\
	
	"Heat capacity" is not the amount of heat a system can take, but rather the amount of heat (energy) needed to raise the system's temperature by one unit (1K). Mathematically, we can write
	$$ C = \frac{dQ}{dT}$$
	Specific heat capacity, on the other hand, is the amount of energy that must be supplied to a unit mass of a substance to raise its temperature by 1K.
	
	There are some complications when we try to measure C of a substance. Gases expand very quickly and while doing so, do some work on the surrounding. This is compensated by requiring extra energy to raise its temperature. Therefore,
	$$ C_p = \left(\frac{\partial Q}{\partial T}\right)_p $$
	is the heat capacity at constant pressure. Similarly,
	\begin{equation*}
	C_V = \left(\frac{\partial Q}{\partial T}\right)_V
	\end{equation*}
	 describes the heat capacity at constant volume.
	 
	 \section{Temperature and Zeroth Law}
	 Two objects, say A and B, are said to be in "thermal contact" if they can exchange energy. When left in contact, heat flows from the hotter body to the colder body spontaneously. When the two attain the same final temperature $T_f$, the flow of heat from A $\rightarrow$ B becomes equal to that from B $\rightarrow$ A.\\
	 The two bodies are then said to be in \textbf{thermal equilibrium} with each other as the net flow of heat becomes zero. \\
	 
	 The \textbf{Zeroth law of thermodynamics} says that if two bodies, A and B are in thermal equilibrium with each other and B is in thermal equilibrium with a third body, say C, then A and C must also be in thermal equilibrium.\\
	 This is a profound fact because it tells us that we can assign a macroscopic property related to energy flow that can be universally compared. We name this property "temperature".\\
	 
	 \subsection{Thermometers}
	 An instrument for measuring temperature can be made. The reading of a thermometer stops changing over time when the object being measured and the thermometer are in thermal equilibrium.\\
	 The first thermometer was made by Galileo which used water. Farenheit invented alcohol as well as mercury thermometers.\\
	 
	 Other instruments include resistance thermometers (which usually use Platinum due to high chemical resistance). Apart from Pt, doped Germaninum which shows high stability over thermal recycling.
	 Carbon sensors, $RuO_2$ (whose resistance goes up on cooling) are all used.\\
	 In cryogenics, the vapour pressure \textendash temperature relationship of liquid $He$ is very useful.
	 
	% \subsection{Problems with thermometers}
	 There are a few challenges with defining temperature using thermometer scales.
	 
	 \begin{enumerate}
	 	\item Firstly, to make very precise measurements, the heat capacity of the thermometer should be \textit{very low}.
	 	
	 	\item No substance gives a completely linear temperature scale. Mercury solidifies at low temperature and vaporizes at greater temperature.
	 	
	 	\item Theoretical calculations can be made using the ideal gas equation for gas but gases liquify at cryogenic temperatures and depart from ideal gas behavior.
	 \end{enumerate}
	 
	\section{Thermodynamic Equilibrium}
	A system is in thermodynamic equilibrium when its macroscopic quantities such as temperature, pressure and volume stop changing over time. Thus, thermodynamic equilibrium requires mechanical, thermal and even chemical equilibrium.\\
	
	When in a state of equilibrium, we can describe the state of the system using \textbf{functions or variables of state}. A relation between these functions can be called an \textbf{equation of state}. Such an equation is of the form.
	
	$$ f(p,V,T) = 0 $$
	
	\subsection{Functions of state}
	To identify which variables can be used to describe the state of a system, we will take advantage of a mathematical property.\\ \textit{Exact differentials} happen to have a property that their definite integral depends only on the initial and final parameters, not the path.
	
	$$ \Delta f = \int_{x_a}^{x_b} df = f(x_b) - f(x_a)$$ 
	
	A variable, or a \textbf{function of state}) which can be used to describe a thermodynamic system is required to have this property. Therefore, a function which can't be represented using an exact differential strictly \textbf{can't} be a function of state.
	
	\section{First law of thermodynamics}
	Lavoiser (1789) came up with the idea of a 'caloric' fluid which hotter objects would have more of than colder objects. However, only nine years later, Rumford remarked that heat could be produced by friction. In the 1840s, Mayer and Joule independently did experiments which rule out the possibility of such a fluid.
	
	\subsection{Joule's experiments}
	Mayer showed the production of heat by friction using a paper pulp. This showed that mechanical work and heat are related.
	
	\subsection{The statement}
	Mayer and Helmholtz suggested a profound statement
	\begin{center}
	\boxed{\textbf{Energy is conserved and heat and work both are forms of energy}}.
	\end{center}
	$$ \Delta U = \Delta Q + \Delta W $$
	where $\Delta Q$ is energy supplied \textbf{to} the system and $\Delta W$ is work done \textbf{on} the system.\\
	
	In differential form,
	
	$$ dU = \dbar Q + \dbar W $$
 
 	The only kind of work done we're considering here is expansion work.
 	
 	$$ \dbar W = -pdV$$
 	where the negative sign assures that when dV is negative (i.e. when the gas is compressed), the work done on the system is positive. Although it should be noted that this relation is strictly true only for reversible processes.

	\subsection{Heat Capacities of Gases}
	Unless controlled, since gases expand upon heating, they do some work which causes them to lose some internal energy.
	
	From the equations above, we find that
	$$ dU = \dbar Q - pdV $$
	
	rearranging this gives
	
	\begin{equation}
	\label{first}
	 \dbar Q = dU + pdV
	\end{equation}

	
	which suggests that when we are heating a gas, a fraction of the energy is used up in doing expansion work. But if the volume is kept constant (which means $\Delta V$ is kept zero), all of the energy supplied can in principle be used up just to raise its internal energy. Thus specific heat would be different if pressure were kept constant and not volume. \\
	
	We now move on to finding quantitatively the difference between the two heat capacities.
	Since internal energy can be defined using just two state functions (say, $U = U(T, V)$), and that internal energy is a state function, we can write its exact differential as
	
	$$ dU = \left(\frac{\partial U}{\partial T}\right)_VdT + \left(\frac{\partial U}{\partial V}\right)_TdV $$ 
	
	Substituting this into equation (\ref{first}) and rearranging gives us
	
	$$ \dbar Q = \left(\frac{\partial U}{\partial T}\right)_V dT + \left(\left(\frac{\partial U}{\partial V}\right)_T + p\right)dV $$
	Dividing by dT on both sides gives us,
	
	$$ \frac{dQ}{dT} = \left(\frac{\partial U}{\partial T}\right)_V + \left(\left(\frac{\partial U}{\partial V}\right)_T + p\right)\frac{dV}{dT}$$
\end{document}