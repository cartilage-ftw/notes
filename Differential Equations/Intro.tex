\documentclass{article}

\usepackage[margin=1.0in]{geometry}

\author{Aayush Arya \\ (tomriddle257@gmail.com)}
\title{Introduction to Ordinary Differential Equations}
\begin{document}
	\maketitle
	
	\section*{Introduction}
	When we solve an algebraic equation $3x + 5 = 8$, we seek a number. The purpose of solving a differential equation such as
	$$ \frac{d^2x}{dt^2} = -\frac{k}{m}x$$
	
	the goal is to find a function x(t) that satisfies it.
	
	A differential equation of a function y(x) has the general form
	$$ f(x, y(x), y'(x), y''(x), ..., y^n(x)) = 0$$
	
	\section*{Nature of Solutions}
	For a differential equation
		$$ y'' + 2y' + 5 = 0$$
		if $y = g(x)$ and $y = h(x)$ are two solutions of the differential equations, then a linear combination of these, i.e.,
		$$ y = c_1g(x) + c_2h(x) $$
		is also a solution.\\
		
	A differential equation of order $n$ has $n$ independent solutions. We can form a general solution of the differential equation using a linear combination of the independent solutions. \textit{Picard's Existence and Uniqueness Theorem} helps us figure out that our general solution is complete and no other solutions are left.\\ But how do we know that we have found a solution? The hallmark sign is that no derivatives of $y$ will be left in the equation.
	The equation will relate $y$ with $x$ at least implicitly.
	
	\section*{Geometric interpretation}
	The geometric interpretation of differential equations is easier to see for linear equations.\\
	A differential equation of the form
	\begin{equation}
	\frac{dy}{dx} = f(x,y)
	\end{equation}
	can be interpreted as a set of vectors pointing towards (1, $\frac{dy}{dx}$) at every point (x,y) in the plane.% \footnote{See Chapter 1 of Simmons-Krantz}
	\\
	Consider the differential equation
	$$ y''(x) = 0 $$
	Solving this gives
	$$ y' = c_1 $$
	$$ y = c_1x + c_2 $$
	Which looks like the familiar equation $y = mx + c$. We note that these arbitrary constants can take up any value. So the equation doesn't represent 'a' straight line but a family of straight lines (of having all possible slope values and y-intercepts).
	
	\section*{Exact Differential Equations}
	An exact or total differential of a function $\psi(x,y)$ is one which shows the contribution of all independent variables for an infinitesimal change
	\begin{equation}
	d\psi = \frac{\partial\psi}{\partial x}dx + \frac{\partial\psi}{\partial y}dy
	\end{equation}
	A differential equation
	$$M(x,y)dx + N(x,y)dy = 0$$ is exact if it can be represented as the exact differential of some function $\psi (x,y)$ where $d\psi (x,y) = 0$
	\\
	That is,
	$$M = \frac{\partial\psi}{\partial x} $$ and $$N = \frac{\partial\psi}{\partial y} $$
	A natural question arises: it would be easy to check the exactness for trivial expressions. Can we write a criterion that assures the exactness of the equation?\\
	
	The equation is exact if
	$$ \frac{\partial M}{\partial y} = \frac{\partial N}{\partial x}$$
	because, if we substitute the expressions for M and Y as partial derivatives of $\psi$, the criterion becomes
	
	$$\frac{\partial^2\psi}{\partial y \partial x} = \frac{\partial^2\psi}{\partial x \partial y}$$
	
	If the equation is exact, it is easy to see that if $d\psi=0$, the solution is just $\psi = constant$
	
	To find $\psi$, we first integrate M(x,y)
	
	$$\psi = \int \frac{\partial \psi}{\partial x}dx + f(y)$$
	The constant of integration is a function of y alone. This can be justified by intuition, because if we take the partial derivative of $\psi$ with respect to x, the terms containing functions of y alone would have vanished. We'd like to bring those back.
	\\
	
	A more rigorous proof for this justification is as follows\\
	
	As we said before,
	$$ N(x,y) = \frac{\partial\psi}{\partial y}$$
	
	substituting the value of $\psi$ from the equation above
	 
	$$ N(x,y) = \frac{\partial\psi}{\partial y} \left(\int \frac{\partial \psi}{\partial x}dx + f(y)\right) $$
	
	taking the partial derivative and transposing for f'(y)
	
	$$ f'(y) = N(x,y) - \frac{\partial\psi}{\partial y} \left(\int \frac{\partial \psi}{\partial x}dx\right) $$
	
	taking the partial derivate with respect to x leaves us with
	
	$$ \frac{\partial f'(y)}{\partial x} = \frac{\partial N}{\partial x} - \frac{\partial^2 }{\partial x \partial y}\left(\int \frac{\partial \psi}{\partial x}dx \right) $$
	
	which on simplifying, becomes
	
	$$ \frac{\partial f'(y)}{\partial x} = \frac{\partial N}{\partial x} - \frac{\partial M}{\partial y} $$
	
	from the condition of exactness, we know that the R.H.S. is zero. Therefore
	
	$$f'(y) = C$$
	
	which means $f(y)$ is some function of y alone.
\end{document}